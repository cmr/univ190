\documentclass[12pt]{article}
\usepackage[utf8]{inputenc}
\usepackage[a4paper,margin=1in]{geometry}
\usepackage{setspace}
\usepackage{parskip}
\usepackage[document]{ragged2e}

\setstretch{1.8}
\title{Poe, Doctor Who, and everything in-between}
\date{\today}
\author{Corey Richardson}

\begin{document}

\setlength{\parindent}{1in}

\maketitle
\thispagestyle{empty}
\newpage
\setcounter{page}{1}

Poe was a very innovative author and invented many genres. A later work by Sir
Arthur Conan-Doyle, \emph{Hound of the Baskervilles}, had elements of both the
modern Horror and Detective genres. It expresses many of the tropes which Poe
invented, including the genius-sidekick relationship and the Hound as the
`nameless horror'. These now-timeless themes pervade even modern television,
as shown by \emph{Doctor Who}.

The most distinctive Poe-ism is the genius/sidekick relationship. In Poe's
detective stories, those featuring C.\ Auguste Dupin, the anonymous narrator
accompanies Dupin throughout many of his adventures. They are introduced as
long-time friends with many common interests. Dupin often asks the narrator
his thoughts on a matter before completely blowing him away with a different
way of thinking. As one example, while walking in silence Dupin makes an
off-hand comment in reply to one of the narrators thoughts. He replies ``this
is beyond my comprehension. I do not hesitate to say that I am amazed, and can
scarcely credit my senses.'' Dupin then explains the exact sequence of
analysis leading to the remark. The narrator is shown as clearly inferior.

Conan-Doyle's characters Holmes and Watson share this relationship as well.
Holmes, of course, being the detective. Their relationship is far more nuanced
than that of Dupin and Poe's narrator, perhaps due to the much lengthier novel
series they are featured in, but perhaps also through Conan-Doyle's evolution
of the role. In particular, Watson takes a more active role as Holmes'
``agent'', not only doing the more uninteresting work but also doing his own
independent investigations while Holmes is pursuing a separate path. Holmes
complements Watson often, in one instance saying he has ``admirable
tenacity''. Even in the face of this, Watson comments internally that ``[He] was
still rather raw over the deception which had been practised upon [him],''
indicating that maybe not all is rosy in their relationship.

This relationship that Poe pioneered continues to this day, very notably in
the \emph{Doctor Who} television series. In this show, the Doctor, a time
traveling humanoid alien known as a Time Lord, travels in his spaceship all
around the galaxy. He is almost always accompanied by a companion, frequently
female, who over time have played varying roles. In recent incarnations, the
companions often serve as foils to the Doctor, and fuel his character
development. This is in stark contrast to Poe and Conan-Doyle's characters. In
the detective stories, the narrators are the closest characters to the
detectives, but even to them the detectives remain mysterious and, as Watson
phrased it, ``self contained''. The detectives remain off-limits, whereas the
Doctor is influenced heavily by his companions. This is not more apparent than
in the most recent series reboot (beginning in 2005), where the Doctor's often
has romantic or near-romantic relationships with his companions.

Another Poe-ism that is now omnipresent is the suspenseful buildup to a big
reveal. This was notably absent from earlier literature, but Poe used this to
great effect in \emph{The Murders in the Rue Morgue}, leading the reader along
for nearly the entire story before the tense confrontation with the sailor.
Dupin gathers many facts that are shared with the reader, but the conclusion
that Dupin correctly reached is entirely unexpected by the reader, due to
clever hiding of key details. During the confrontation scene, Dupin lays out
the entire story for the reader. Poe also uses this technique in \emph{The
Purloined Letter}, much to the same effect. Conan-Doyle did not make such
fantastic jumps as Poe when doing his reveal, rather choosing to make all of
Holmes' knowledge known before the conflict with Stapleton. The suspense comes
from the tense chase scene with the hound, especially from not knowing whether
Holmes will actually succeed in protecting Sir Henry from the beast. In Doctor
Who, entire episodes are often dedicated to solving a particular mystery. In
\emph{Unicorn and the Wasp}, Agatha Christie, a murder mystery novelist,
serves as a character to help the Doctor unravel an ongoing murder mystery.
However, suspense is now a feature of almost every work of fiction. This is
not an obvious evolution. Older works, such as the Holy Bible or the Odyssey,
feature prophecies which determine the entire plot from the start.
Contemporaries of Poe also began to use this technique, such as Hawthorne
keeping the secret of Pearl's father hidden for a large portion of the Scarlet
Letter.

Poe also innovated on the horror genre. Taking its influence from earlier
Gothic literature, Poe wrote dark poems such as \emph{The Raven}, which
features the subject going mad. Longer stories such as \emph{The Fall of the
House of Usher}, \emph{The Tell-tale Heart}, \emph{The Pit and the Pendulum},
and \emph{The Cask of Amontillado} all have dark themes. Of most importance is
the sort of `nameless horror' that H.P.  Lovecraft would later master. This
features most strongly in The Pit and the Pendulum, where the contents of the
pit are never revealed, but very much feared. This feeds on the very human
fear of the unknown, as well as the imagination filling in missing details, to
create a sense of dread or shock. In Doctor Who, \emph{Horror of Fang Rock}
serves as the best example. The Doctor crash-lands on a foggy island, near a
lighthouse. As the episode progresses, an unidentified force begins killing
off the inhabitants of the island. It is eventually revealed to be a hostile
alien, but the foggy scene and large portion of the episode where the enemy is
unknown serve to instill fear in the audience.

\end{document}
